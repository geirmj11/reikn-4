\documentclass[11pt]{article}
\usepackage[icelandic]{babel}
\usepackage{amsmath}
\usepackage{array}
\usepackage[margin=1in]{geometry}            
\geometry{a4paper} 
\setlength{\parindent}{0mm}
\usepackage[T1]{fontenc}
\usepackage{fourier}
\usepackage[condensed]{kurier}
%fyrir myndir
\usepackage[pdftex]{graphicx}
\usepackage[table]{xcolor}

\usepackage{pgf}
\usepackage[utf8]{inputenc}
\usepackage{listings}

\pagestyle{headings}

\usepackage{fancyhdr}
\setlength{\headheight}{25.5pt}
\pagestyle{fancy}

\title{Programming Assignment 4}
\author{}
\date{}

\lhead[lh-even]{T-301-REIR - Fall 2012 \\ Programming Assignment #4}
\rhead[rh-even]{Geir Matti Järvelä (geirmj11@ru.is) \\ Þorgeir Auðunn Karlsson (thorgeirk11@ru.is)}

\begin{document}
\section*{\centering Programming Assignment #4: Properties of Social Networks}
For the assignment we created three classes, explained further below. All of the classes implement an intelligent lazy solution to each problem.
In effect this means that all calculations are done at most once, and only when they are needed.
Each method checks first if a solution exists, and if so simply returns a previously calculated solution. This method costs a little bit
more in memory, since previous solutions need to be kept, but makes up for it with constant calculating times when duplicating previous calculations, or when 
two functions use the same results.

\subsubsection*{Part I - The Centrality class}
Degree: The degree() method counts the number of edges to a given Vertex, and returns that number. \\
Popularity: The popularVertex() method calculates the degree for all vertices and keeps track of the highest degree, returning the corresponding vertex. \\
Eccentricity: The ecc() method creates a BreathFirstPath for the given node, and then finds the highest distance to every other node, and returns the highest distance. \\
Center: The center() method goes through the array of eccentrities, calls ecc() for every node that has not been calculated before, and finds the lowest value.\\
Closeness: The closeness() method creates a BreathFirstPath for the given node, goes through all the paths in the resulting collection and adds their distance 
to the hollywood variable. It then divides one with that variable, and returns the resulting value.\\
Closest: The closest() method calculates the closeness of all vertices, keeping track of the highest value, and returns the vertex with that value.\\
Effective Eccentricity: The effEcc() method generates an arrayList and creates a breathFirstPath for the given node, it then adds all distances from 
the BreadthFirstPath to the list, sorts the resulting list, and returns the value in the 90\% position in the sorted list. \\
Effective Center: The effCenter() method goes through the list of effective eccentricities and generates the missing values, and returns the node with the lowest value.\\ \\

\begin{minipage}[b]{0.5\linewidth}
The numbers in the table on the right show the complexity of each method, for the first time each method is called. 
However some of the methods call other methods for every point, for example the closest() method calls the closeness() function for every point in the graph,
and since the results are stored, they can be accessed directly thereafter. \\ \\ 
\end{minipage}
\begin{minipage}[b]{0.5\linewidth}\centering
\scalebox{0.7}{
\begin{tabular}{l|cc}
    Method & Initial Complexity & Subsequent\\
	\hline
	 Constructor            & $10$                  & constant\\
	 Degree                 & $\sim V$              & constant\\
	 Eccentricity           & $\sim2V+E$            & constant\\
	 Effective Eccentricity & $\sim V(\log V)$      & constant\\ %3V+E+V(log V)
	 Closeness              & $\sim2V+E$            & constant\\
	 Most Popular Vertex    & $\sim V^2$            & constant\\
	 Center                 & $\sim V$              & constant\\
	 Effective Center       & $\sim V(V(\log V)) $  & constant\\
	 Closest Vertex         & $\sim V(V+E)$         & constant\\
	 \hline
\end{tabular}
}
\small{Complexity table: $V$: Number of Vertices \\$E$: Number of Edges.} \\ \\ 
\end{minipage}

\subsubsection*{Part II - The SymbolCentrality class}
The SymbolCentrality class is simply an extention of Centrality. It creates a Symbolgraph class instance, and uses it to create indexed integer keys for each actor.\\
This allows the class to use the methods of the Centrality class to calculate all necessary values. \\ \\
Since each function calls the sg.index() method, this adds log(n) complexity to each operation of the Centrality class. 
In addition more memory is needed because of the SymbolGraph that stores the indices.

\subsubsection*{Part III - The ExtendedBreadthFirstPaths class}
Here we extend the Breath First Search. The generate() method creates a queue and initializes the array numOfPaths[] with a value of -1. Each field
represents a single vertex in the graph. The base path is initialized to one, since the base node has one path to itself. The base node is then added
to the queue, and given a level of zero, in the levels[] array. Each node connected to base is is given a level of one higher, and the number of paths to 
zero, in order to indicate that they have been added to the queue. We then go through each node on the queue, take the number of paths of each node connected to
them that have already been marked, and add that to the number of shortest paths for the current node. We then give all adjacent nodes that have not been marked 
a level of one higher, and add it to the queue. At the same time we keep track of the vertex with the highest number of paths, and once the method has gone through
all vertices, we return that value. \\
The complexity of this method is $\sim2V+E$ where $V$ is the number of vertices and $E$ is the number of edges.

\end{document} 

